\documentclass[8pt]{beamer}

\newif\ifplacelogo % create a new conditional
\placelogotrue % set it to true

\usetheme{Warsaw}
\usecolortheme{rose}
\usepackage{multicol}
\usepackage{epstopdf}
\usepackage[italic]{hepnames}
\usepackage{tikz}
\usepackage{listings}
\usepackage{times}
\usepackage{amsmath}
\usepackage{verbatim}
\usepackage{hyperref}
\usepackage{bbding}
\lstset{breakatwhitespace,
language=C++,
columns=fullflexible,
keepspaces,
breaklines,
tabsize=3, 
showstringspaces=false,
extendedchars=true}

% TikZ includes!!!
\usepackage{tikz}
\usetikzlibrary{backgrounds}
\usetikzlibrary{calc}
\tikzstyle{every picture}+=[remember picture]
\input{/home/oviazlo/PhD_study/myReports/latexHelpScripts/tikzGrid.tex}


\begin{document}


\graphicspath{ {/home/oviazlo/PhD_study/myReports/pictures/Wprime/DIS2016/} }
\DeclareGraphicsExtensions{.eps, .pdf, .png}

% poster advertisement
\newcommand{\myCenterBox}[2][pink] {
   {\centering
    \noindent\colorbox{#1}{
	\textbf{#2}
    }\par
  }
}

\newcommand{\mySmallCenterBox}[2][pink] {
   {\centering
    \noindent\colorbox{#1}{
	\textbf{{\small #2}}
    }\par
  }
}

\newcommand{\myVerySmallCenterBox}[2][pink] {
   {\centering
    \noindent\colorbox{#1}{
	\textbf{{\scriptsize #2}}
    }\par
  }
}

\newcommand{\backupbegin}{
   \newcounter{finalframe}
   \setcounter{finalframe}{\value{framenumber}}
}
\newcommand{\backupend}{
   \setcounter{framenumber}{\value{finalframe}}
}

\newcommand{\myNode}{\tikz[baseline,inner sep=1pt] \node[anchor=base]}

\definecolor{light-gray}{gray}{0.95}
% poster advertisement


\title[ Searches with high-mass fermionic final states and jets \hspace{6.5em}\insertframenumber/
\inserttotalframenumber]{ Searches for new physics in high-mass fermionic final states and jets with the ATLAS detector at the LHC }


	\author[Oleksandr Viazlo on behalf of the ATLAS collaboration]{Oleksandr Viazlo \\ {\small on behalf of the ATLAS collaboration}}
	\institute{\small Lund University\\} 
	
       
	\date{13 April 2016}

	\logo{ \ifplacelogo \includegraphics[height=1.8cm]{./ID_week2/lund_uni-logo_s.pdf} \hspace{0.4cm} \fi}

	
   	\frame{\titlepage}

   	

\placelogofalse

%*****************************************************************************
\begin{frame}{\large \large Introduction}
 
 \myCenterBox{Resonant and non-resonant searches for new physics}
 
 
 \begin{columns}
  \begin{column}{6.5cm}
  \vspace{0.25cm} \\
  \noindent\colorbox{pink}{
	\textbf{Covered analysis}
    }\par
    \vspace{0.1cm}
    \begin{itemize}
     \item 13 TeV data only.
     \begin{itemize}
      \item 3.0 - 3.6 $fb^{-1}$ of luminosity.
     \end{itemize}
%       \vspace{0.2cm}
     \item High-mass final states only.
    \end{itemize}
  
    \includegraphics[width=6cm]{intlumivstime2015DQ.pdf}\\
    
    \begin{itemize}
     \item 8 analysis in total $\to$ no many details on the slides $\to$ follow references to find out more!
    \end{itemize}

    
    \begin{tikzpicture}[overlay]

    %% HELPER draw advanced helping grid with axises:
%     \draw (0,0.5) to[grid with coordinates] (6,9);

    \end{tikzpicture}
  
  \end{column}
  \begin{column}{6cm}
  
      
   \noindent\colorbox{pink}{
	\textbf{Fermionic final states}
    }\par
   \begin{itemize}
    \item {\bf lepton + $E_\mathrm{T}^{miss}$} [$\PWprime$]
    \item {\bf di-lepton $\Plepton^{\pm}\Plepton^{\mp}$} [$\PZprime$, Contact Interaction (CI)]
    \item {\bf di-lepton $e^{\pm}\mu^{\mp}$} [$\PZprime$, Quantum Black Holes (QBH)]
   \end{itemize}
   \vspace{0.5cm}
   \noindent\colorbox{pink}{
	\textbf{Hadronic final states (jets)}
    }\par
   \begin{itemize}
    \item {\bf di-jet} [QBH, $q^{*}$, $\PWprime$, CI]
    \item {\bf di-jet with b-tagging} [$b^{*}$, $\PZprime$]
    \item {\bf multijet} [QBH, string balls]
   \end{itemize}
   \vspace{0.5cm}
   \noindent\colorbox{pink}{
	\textbf{Mixed final states}
    }\par
    \begin{itemize}
    \item {\bf lepton and 2 jets/leptons} [QBH]
    \item {\bf 2 leptons and 2+ jets} [leptoquarks]
   \end{itemize}
    
  \end{column}
 \end{columns}
 
\end{frame}
%*****************************************************************************
%*****************************************************************************
\begin{frame}{\large Physics objects performance at 13 TeV}

 \begin{columns}
  \begin{column}{4cm}
   \myVerySmallCenterBox{Muon Identification Efficiencies}
   \includegraphics[width=4cm]{performance/muEff.pdf}\\
  \end{column}
  \begin{column}{4cm}
   \myVerySmallCenterBox{$E_\mathrm{T}^{miss}$ uncertainty}
   \includegraphics[width=4cm]{performance/MET_rms_muons.png}\\
  \end{column}
  \begin{column}{4cm}
   \begin{itemize}
    \item \textbf{Lepton} efficiencies and energy corrections are measured in $J/\psi \to \Plepton\Plepton$
    and $Z \to \Plepton\Plepton$ events.
    \item \textbf{$E_\mathrm{T}^{miss}$} resolution and scale are evaluated from $Z\to \mu\mu$ events.
   \end{itemize}
  \end{column}
 \end{columns}

 \vspace{0.5cm}
 
 \begin{columns}
  \begin{column}{4cm}
   \myVerySmallCenterBox{Jet Energy Scale uncertainty}
   \includegraphics[width=4cm]{performance/fig_06.png}\\
  \end{column}
  \begin{column}{4cm}
   \myVerySmallCenterBox{Light jet rejection vs. b-jet tag efficiency}
   \includegraphics[width=4cm]{performance/bTaggingEff.png}\\
  \end{column}
  \begin{column}{4cm}
   \begin{itemize}
    \item \textbf{Jet} Energy Scale uncertainty is below 3$\%$ for TeV jets.
    \item Significant improvement of \textbf{b-tagging} performance:
    \begin{itemize}
    {\small
     \item Insertable B-Layer (IBL).
     \item better tracking in dense environment.
     \item improved b-tagging algorithms.
     }
    \end{itemize}

   \end{itemize}
  \end{column}
 \end{columns}
 
 
  \begin{tikzpicture}[overlay]

  %% HELPER draw advanced helping grid with axises:
%     \draw (0,0) to[grid with coordinates] (6,9);
  
    \node[right] (textNode) at (0.45,5.0) {
      {\tiny \href{http://arxiv.org/abs/1603.05598}{arXiv:1603.05598} }
    };
    
    \node[right] (textNode) at (4.7,4.8) {
      {\tiny \href{http://atlas.web.cern.ch/Atlas/GROUPS/PHYSICS/PLOTS/JETM-2016-003/}{JETM-2016-003} }
    };

    \node[right] (textNode) at (0.45,0.5) {
      {\tiny \href{http://cds.cern.ch/record/2037613}{ATL-PHYS-PUB-2015-015} }
    };

    \node[right] (textNode) at (4.5,0.0) {
      {\tiny \href{http://cds.cern.ch/record/2037697}{ATL-PHYS-PUB-2015-022} }
    };
    
    
    
  \end{tikzpicture}
 
\end{frame}
%*****************************************************************************
%*****************************************************************************
% \bgroup
% \setbeamercolor{background canvas}{bg=white}
\begin{frame}{}

    \begin{tikzpicture}[overlay]

    %% HELPER draw advanced helping grid with axises:
%     \draw (0,-5) to[grid with coordinates] (11,3);

    \node[right] (textNode) at (3,0) {
      { \large \bf Fermionic final states  }
    };
    
    \node[right] (n1) at (5,-0.7) {
        \EightStarTaper lepton + $E_\mathrm{T}^{miss}$
    };
    
    \node[right] (n2) at (5,-1.2) {
        \EightStarTaper di-lepton $\Plepton^{\pm}\Plepton^{\mp}$
    };
    
    \node[right] (n3) at (5,-1.7) {
        \EightStarTaper di-lepton $e^{\pm}\mu^{\mp}$
    };
    
    \tikz[overlay]\draw[thick,black,->] ([xshift=-0.5cm]textNode.south) to [out=270, in=180] ([xshift=-0.1pt]n1.west);
    \tikz[overlay]\draw[thick,black,->] ([xshift=-1.0cm]textNode.south) to [out=270, in=180] ([xshift=-0.1pt]n2.west);
    \tikz[overlay]\draw[thick,black,->] ([xshift=-1.5cm]textNode.south) to [out=270, in=180] ([xshift=-0.1pt]n3.west);

    \end{tikzpicture}

\end{frame}
% \egroup
%*****************************************************************************
%*****************************************************************************
\begin{frame}{\large lepton + $E_\mathrm{T}^{miss}$}
  
  \begin{columns}
   \begin{column}{6cm}
    \myCenterBox{Electron channel}
    \includegraphics[width=6cm]{leptonPlusMET/fig_01a.png}\\
   \end{column}
   \begin{column}{6cm}
    \myCenterBox{Muon channel}
    \includegraphics[width=6cm]{leptonPlusMET/fig_01b.png}\\
   \end{column}
  \end{columns}
  \begin{columns}
   \begin{column}{6cm}
    \begin{itemize}
     \item Exactly \textbf{one isolated electron(muon)} with $p_\mathrm{T}>$ 65(55) GeV
     \item \textbf{$E_\mathrm{T}^{miss}$} $>$ 65(55) GeV
     \item Search variable: \\ $m_\mathrm{T} = \sqrt{2 p_\mathrm{T} E_\mathrm{T}^{miss} (1-\cos\varphi_{\Plepton\nu})}$
    \end{itemize}
    \myCenterBox[yellow]{\href{https://cds.cern.ch/record/2114829}{ATLAS-CONF-2015-063}}
   \end{column}
   \begin{column}{6cm}
    \begin{itemize}
     \item Dominant background processes:
     \begin{itemize}
       \item W, Z - from Monte Carlo simulation\\ {\scriptsize (mass dependent NNLO pQCD and NLO EW corrections)}
       \item top - from Monte Carlo simulation\\ {\scriptsize (accurate to NNLO in pQCD)}
       \item multijet - data-driven “matrix” method
     \end{itemize}
     \item No significant deviation from SM is observed.
    \end{itemize}
   
   \end{column}
  \end{columns}

  
 \end{frame}
%*****************************************************************************
%*****************************************************************************
  \begin{frame}{\large lepton + $E_\mathrm{T}^{miss}$}
  
  \begin{columns}
   \begin{column}{5cm}
   
    \myCenterBox{13 TeV results}
    {\centering
    \includegraphics[width=5cm]{leptonPlusMET/tab_02.pdf}\\
    }
   
    \myCenterBox{8 TeV results}
%       \begin{table}[!htbp]
% 	\centering
	\begin{tabular}{c|cc}
	  \hline
	  \hline
	  Decay     &  ATLAS & CMS \\
	  \hline
	  \ensuremath{\PWprime\rightarrow \ell \nu} & 3.24 TeV & 3.28 TeV \\
	  \hline
	  \hline
	\end{tabular}
%       \end{table}
   \vspace{0.2cm}\\
   \myCenterBox[yellow]{\href{https://cds.cern.ch/record/2114829}{ATLAS-CONF-2015-063}}
   \end{column}
   \begin{column}{7cm}
    \includegraphics[width=7cm]{leptonPlusMET/fig_03.png}\\
    
   \end{column}
  \end{columns}
  
  \begin{itemize}
    \item Benchmark model: $\PWprime$ Sequential Standard Model (SSM)
      \begin{itemize}
	\item Heavier copy of the SM W, same couplings
	\item Branching ratio to WZ set to 0, no interference to W
      \end{itemize}
    \item {\bf Almost 1 TeV improvement on limits}
  \end{itemize}
  
 \end{frame}
%*****************************************************************************
%*****************************************************************************
 \begin{frame}{\large di-lepton $\Plepton^{\pm}\Plepton^{\mp}$}
    \begin{columns}
   \begin{column}{6cm}
    \myCenterBox{Di-electron}
    \includegraphics[width=6cm]{dilepton/fig_01a.png}\\
   \end{column}
   \begin{column}{6cm}
    \myCenterBox{Di-muon}
    \includegraphics[width=6cm]{dilepton/fig_01b.png}\\
   \end{column}
  \end{columns}
  \begin{columns}
   \begin{column}{6cm}
    \begin{itemize}
     \item At least \textbf{2 isolated leptons} \\with $p_\mathrm{T} >$ 30 GeV
     \vspace{0.2cm}
     \item Search variable: di-lepton invariant mass
    \end{itemize}
    \myCenterBox[yellow]{\href{https://cds.cern.ch/record/2114842}{ATLAS-CONF-2015-070}}
   \end{column}
   \begin{column}{6cm}
    \begin{itemize}
     \item Dominant background processes:
     \begin{itemize}
      \item Drell-Yan Z/$\gamma$ - from Monte Carlo simulation\\ {\scriptsize (mass dependent NNLO pQCD and NLO EW corrections)}
     \end{itemize}
%      \item MC are rescaled in the normalisation region 80 GeV $< m_{\Plepton\Plepton} <$ 120 GeV.
     \item Sums of backgrounds are rescaled to match data 
	   in the normalisation region \\ 80 GeV $< m_{\Plepton\Plepton} <$ 120 GeV.
     \item No significant excess observed.
    \end{itemize}
     
   \end{column}
  \end{columns}
  
 \end{frame}
%*****************************************************************************
%*****************************************************************************
\begin{frame}{\large di-lepton $\Plepton^{\pm}\Plepton^{\mp}$}

\begin{columns}
  \begin{column}{5.5cm}
  
  \myCenterBox{\textbf{Resonant} $\PZprime$ models}
  \begin{itemize}
    \item Benchmark SSM (same as for the $\PWprime$):
    \begin{itemize}
      \item $\PZprime$ has the same couplings as SM Z
    \end{itemize}
    \item $E_6$ Grand Unified Theory:
    \begin{itemize}
      \item $E_6 \rightarrow SU(5) \times U(1)_{\chi} \times U(1)_{\psi}$
      \item $\PZprime(\theta_{E_6}) = Z'_{\psi}\cos\theta_{E_6} + Z'_{\chi}\sin\theta_{E_6}$
      \item 6 commonly motivated states of $\PZprime$ namely:
	    $Z'_{\psi}, Z'_{N}, Z'_{\eta}, Z'_{I}, Z'_{S}, Z'_{\chi}$
    \end{itemize}
    \item {\bf 400-500 GeV improvements on limits over Run 1 results}
  \end{itemize}
  
  \myCenterBox[yellow]{\href{https://cds.cern.ch/record/2114842}{ATLAS-CONF-2015-070}}
  
  \end{column}
  \begin{column}{7cm}
  \includegraphics[width=7cm]{dilepton/fig_03c.png}\\
  
  \end{column}
\end{columns}

{\centering
\includegraphics[width=8cm]{dilepton/tab_03.pdf}\\
}
% \begin{columns}
%  \begin{column}{7cm}
%   \includegraphics[width=7cm]{dilepton/tab_03.pdf}\\
%  \end{column}
%  \begin{column}{5cm}
%   \includegraphics[width=5cm]{dilepton/fig_02c.png}\\
%  \end{column}
% \end{columns}


\end{frame}
%*****************************************************************************
%*****************************************************************************
\begin{frame}{\large di-lepton $\Plepton^{\pm}\Plepton^{\mp}$}

\begin{columns}
  \begin{column}{5cm}
  
  \myCenterBox{\textbf{Non-Resonant} Contact Interaction (CI)}
  \begin{itemize}
    \item Quark and Lepton Compositeness:
    \begin{itemize}
      \item Characteristic energy scale $\varLambda$ corresponds to binding energy between constituents
      \item $\eta_{ij}$ gives chiral structure of the interaction
    \end{itemize}
    \item Excpect non-resonant deviations in the di-lepton mass spectrum.
  \end{itemize}
  \myCenterBox[yellow]{\href{https://cds.cern.ch/record/2114842}{ATLAS-CONF-2015-070}}
  \end{column}
  \begin{column}{7cm}
  \includegraphics[width=7cm]{dilepton/fig_04.png}\\
  
  \end{column}
\end{columns}

{\centering
  \includegraphics[width=7cm]{dilepton/CI_feynman.png}\\
}

\end{frame}
%*****************************************************************************
%*****************************************************************************
\begin{frame}{\large di-lepton $e^{\pm}\mu^{\mp}$}

\begin{columns}
  \begin{column}{6cm}
    \begin{itemize}
      \item Direct production of $e^{\pm}\mu^{\mp}$ pair is forbidden by 
	    lepton flavour conservation in SM.\\ 
      \item Is allowed in many extensions of the SM \\
	    ($\PZprime$, Quantum Black Hole (QBH) models).
      \item Main backgrounds - from MC simulations. \\
	    Multi-Jet - data-driven matrix method.
      \item \textbf{0.5 TeV improvement on $\PZprime$ limit.}
    \end{itemize}
    
    \includegraphics[width=5.8cm]{emuDilepton/fig_08a.png}\\
    
  \end{column}
  \begin{column}{6.2cm}
    \myCenterBox[yellow]{\href{https://cds.cern.ch/record/2114844}{ATLAS-CONF-2015-072}}    
    \includegraphics[width=6.5cm]{emuDilepton/fig_07a.png}\\
    
%     \myCenterBox{$\PZprime$} \\

%     \includegraphics[width=5.8cm]{emuDilepton/fig_08a.png}\\
%     \myCenterBox{QBH} \\
%     \includegraphics[width=5.8cm]{emuDilepton/fig_08b.png}\\
      
  \end{column}
\end{columns}

{\centering
  \includegraphics[width=8cm]{emuDilepton/tab_03.png}\\
}

\end{frame} 
%*****************************************************************************
%*****************************************************************************
% \bgroup
% \setbeamercolor{background canvas}{bg=white}
\begin{frame}{}

    \begin{tikzpicture}[overlay]

    %% HELPER draw advanced helping grid with axises:
%     \draw (0,-5) to[grid with coordinates] (11,3);

    \node[right] (textNode) at (3,0) {
      { \large \bf Hadronic final states (jets) }
    };
    
    \node[right] (n4) at (5,-0.7) {
        \EightStarTaper di-jet
    };
    
    \node[right] (n5) at (5,-1.2) {
        \EightStarTaper di-jet with b-tagging
    };
    
    \node[right] (n6) at (5,-1.7) {
        \EightStarTaper multijet
    };
    
    \tikz[overlay]\draw[thick,black,->] ([xshift=-0.8cm]textNode.south) to [out=270, in=180] ([xshift=-0.1pt]n4.west);
    \tikz[overlay]\draw[thick,black,->] ([xshift=-1.3cm]textNode.south) to [out=270, in=180] ([xshift=-0.1pt]n5.west);
    \tikz[overlay]\draw[thick,black,->] ([xshift=-1.8cm]textNode.south) to [out=270, in=180] ([xshift=-0.1pt]n6.west);

    \end{tikzpicture}

\end{frame}
% \egroup
%*****************************************************************************
%*****************************************************************************
\begin{frame}{\large Di-jet}
\begin{columns}
 \begin{column}{6cm}
    \begin{itemize}
      \item At least \textbf{two jets} with $p_\mathrm{T}>$ 440 (50) GeV.
      \vspace{0.2cm}
      \item Search variable: mass $m_{jj}$ of the two leading jets.
      \vspace{0.2cm}
      \item QCD predicts a smoothly falling dijet mass $\rightarrow$ model it with power law function: \\
      \vspace{0.1cm}
      $f(z)=p_{1}(1-z)^{p_{2}}z^{p_{3}}$, where $z \equiv m_{jj}/\sqrt{s}$
      \vspace{0.2cm}
      \item Look for localized excesses wrt. background model.
      \vspace{0.2cm}
      \item Di-jet mass resolution: 2-2.5$\%$
      \vspace{0.2cm}
      \item No significant deviation observed.
    \end{itemize}
 \end{column}
 \begin{column}{6cm}
  \includegraphics[width=6cm]{dijet/fig_01.png}\\
  \myCenterBox[yellow]{\href{http://arxiv.org/abs/1512.01530}{arXiv:1512.01530}}
 \end{column}
\end{columns}
\end{frame}
%*****************************************************************************
%*****************************************************************************
\begin{frame}{\large Di-jet angular analysis}
 \begin{columns}
 \begin{column}{6cm}
    \begin{itemize}
      \item Complementary to di-jet resonance analysis which does not have sensitivity for wide resonance.
      \vspace{0.2cm}
      \item Search variable: invariant rapidity $\chi$ in different $m_{jj}$ ranges.
      \vspace{1.5cm}
      \item Beyond Standard Model (BSM) signal are expected at large angles wrt. the beam $\rightarrow$ at low $\chi$.
      \vspace{0.2cm}
%       \item The highest $m_{jj}$ measured is 7.9 TeV.
      \item No significant excess observed.
      \vspace{0.2cm}
      \item New mass region explored: $m_{jj}>5.4$ TeV was not reachable for 8 TeV analysis.
    \end{itemize}
    \vspace{1.7cm}
    \begin{tikzpicture}[overlay]

    %% HELPER draw advanced helping grid with axises:
%     \draw (0.5,-2.5) to[grid with coordinates] (6,10);
    
     \node[right] (textNode) at (0,0.9) {
       \includegraphics[width=2.9cm]{chi_1.png}
     };
     \node[right] (textNode) at (3,0.9) {
       \includegraphics[width=2.9cm]{chi_14.png}
     };
     \node[right] (textNode) at (0.5,5.8) {
       \mySmallCenterBox[light-gray]{$y = ln\left(\dfrac{E+p_z}{E-p_z}\right) \rightarrow y^{*} = \dfrac{y1-y2}{2}$}
     };
     \node[right] (textNode) at (1.7,5.05) {
       \mySmallCenterBox[light-gray]{$\chi = e^{2|y^{*}|} = e^{|\Delta y|}$}
     };
 
    \end{tikzpicture}
    
 \end{column}
 \begin{column}{6cm}
  \includegraphics[width=6cm]{dijet/fig_02.png}\\
  \myCenterBox[yellow]{\href{http://arxiv.org/abs/1512.01530}{arXiv:1512.01530}}
 \end{column}
\end{columns}
\end{frame}
%*****************************************************************************
%*****************************************************************************
\begin{frame}{\large Di-jet}
\begin{columns}
 \begin{column}{6cm}
  \includegraphics[width=5.5cm]{dijet/fig_03.png}\\
 \end{column}
 \begin{column}{6cm}
  \includegraphics[width=5cm]{dijet/fig_06.png}\\
  \myCenterBox[yellow]{\href{http://arxiv.org/abs/1512.01530}{arXiv:1512.01530}}
 \end{column}
\end{columns}
{\centering
\includegraphics[width=9cm]{dijet/tab_01.png}\\
}
\end{frame}
%*****************************************************************************
%*****************************************************************************
\begin{frame}{\large Di-jet with b-tagging}

\begin{columns}
  \begin{column}{6cm}
    \begin{itemize}
      \item Similar selection as for di-jet analysis. 
      \vspace{0.15cm}
      \item Signal regions:
      \begin{itemize}
       \item $\geq$1 b-tagged jets.
       \item 2 b-tagged jets.
      \end{itemize}
      \vspace{0.15cm}
      \item Background prediction is compatible with data.
      \vspace{0.15cm}
      \item Excluded mass limits: \\
        \begin{itemize}
         \item \textbf{$b^*$: 1.1-2.1 TeV}
         \item \textbf{leptophobic $\PZprime$: 1.1-1.5 TeV}
        \end{itemize}
      \vspace{0.15cm}
%       \item \textbf{Original 13 TeV limits.}
      \item No such analysis for 8 TeV period.
    \end{itemize}
%     \vspace{1cm}
    {\centering
      \includegraphics[width=6cm]{bTaggedDijet/fig_05.png}\\
    }
  \end{column}
  \begin{column}{6cm}
    {\centering
    \includegraphics[width=4.5cm]{bTaggedDijet/excitedQuark.png}\\
    }
    \vspace{0.2cm} 
    \includegraphics[width=6cm]{bTaggedDijet/fig_04a.png}\\
    \myCenterBox[yellow]{\href{http://arxiv.org/abs/1603.08791}{arXiv:1603.08791}}
  \end{column}
\end{columns}

\end{frame} 
%*****************************************************************************
%*****************************************************************************
\begin{frame}{\large Multijet}
 \begin{itemize}
  \item At least \textbf{three jets} with $p_\mathrm{T} >$ 50 GeV and $|\eta| <$ 2.8
  \vspace{0.2cm}
  \item Search variable: Scalar sum of jet transverse momenta ($H_\mathrm{T}$) $>$ 1 TeV
  \vspace{0.2cm}
  \item SM background prediction - smooth fit by data (10 functions tested):
  \begin{itemize}
   \item is done in low-$H_\mathrm{T}$ Control Region (CR)
   \item is cross-checked in medium-$H_\mathrm{T}$ Validation Region (VR)
   \item is used for SM backgound prediction in high-$H_\mathrm{T}$ Signal Region (SR)
  \end{itemize}
  \vspace{0.2cm}
  \item No significant excess observed.
%   \item In order to be sure that CR and VR are not contaminated by possible signal - 
% 	use bootstrap approach: re-define regions and search in incrementally larger data sets
 \end{itemize}
 \vspace{0.2cm}
 \begin{columns}
  \begin{column}{7cm}
%    \myCenterBox{Step 1: 6.5 pb$^{-1}$}
   {\centering
   \includegraphics[width=7cm]{multijet/figaux_17a.png}\\
   }
  \end{column}
%   \begin{column}{3cm}
%    \myCenterBox{Step 2: 74 pb$^{-1}$}
%    \myCenterBox{Step 3: 0.44 fb$^{-1}$}
%   \end{column}
  \begin{column}{5cm}
%   \myCenterBox{Step 4: 3.0 fb$^{-1}$}
  {\centering
   \includegraphics[width=5cm]{multijet/fig_08.png}\\
   }
  \end{column}
 \end{columns}

  \begin{tikzpicture}[overlay]

  %% HELPER draw advanced helping grid with axises:
%       \draw (0,0) to[grid with coordinates] (12,6);

  \node[right] (textNode) at (7.9,4.9) {
      \myVerySmallCenterBox{CR}
  };
  \node[right] (textNode) at (9.05,4.9) {
      \myVerySmallCenterBox{VR}
  };
  \node[right] (textNode) at (10.1,4.9) {
      \myVerySmallCenterBox{SR}
  };
  \node[right] (textNode) at (8,5.5) {
      \myCenterBox[yellow]{\href{http://arxiv.org/abs/1512.02586}{arXiv:1512.02586}}
  };
  
  
  \end{tikzpicture}
 
\end{frame}
%*****************************************************************************
%*****************************************************************************
\begin{frame}{\large Multijet}
  \begin{columns}
  \begin{column}{6cm}
   \myCenterBox{Rotating black holes model}
   \includegraphics[width=6cm]{multijet/fig_09.png}\\
   
  \end{column}
  \begin{column}{6cm}
   \myCenterBox{String ball model}
   \includegraphics[width=6cm]{multijet/fig_11b.png}\\
   \end{column}
 \end{columns}
 

  \begin{columns}
  \begin{column}{6cm}
    \begin{itemize}
      \item Limits for rotating black holes with \\2,4 and 6 extra dimensions are set  as a function:
      \begin{itemize}
      \item fundamental Planck scale ($M_D$)
      \item mass threshold ($M_{th}$)
      \end{itemize}
    \end{itemize}
    
  \end{column}
  \begin{column}{6cm}
    \begin{itemize}
      \item Limits for String balls with 6 extra dimensions are set as a function:
      \begin{itemize}
      \item string scale ($M_S$)
      \item mass threshold ($M_{th}$)
      \item string coupling ($g_S$)
      \end{itemize}
    \end{itemize}
  \end{column}
 \end{columns}
 \vspace{0.5cm}
 {\centering
 \textbf{Almost 4 TeV improvement of limit in $M_{th}$}.\\
 }
 \myCenterBox[yellow]{\href{http://arxiv.org/abs/1512.02586}{arXiv:1512.02586}}
 
\end{frame}
%*****************************************************************************
%*****************************************************************************
% \bgroup
% \setbeamercolor{background canvas}{bg=white}
\begin{frame}{}

    \begin{tikzpicture}[overlay]

    %% HELPER draw advanced helping grid with axises:
%     \draw (0,-5) to[grid with coordinates] (11,3);

    \node[right] (textNode) at (3,0) {
      { \large \bf Mixed final states }
    };
    
    \node[right] (n7) at (4.4,-0.7) {
        \EightStarTaper lepton and 2 jets/leptons
    };
    
    \node[right] (n8) at (4.4,-1.2) {
        \EightStarTaper 2 leptons and 2+ jets
    };
        
    \tikz[overlay]\draw[thick,black,->] ([xshift=-0.8cm]textNode.south) to [out=270, in=180] ([xshift=-0.1pt]n7.west);
    \tikz[overlay]\draw[thick,black,->] ([xshift=-1.3cm]textNode.south) to [out=270, in=180] ([xshift=-0.1pt]n8.west);
    

    \end{tikzpicture}

\end{frame}
% \egroup
%*****************************************************************************
%*****************************************************************************
\begin{frame}{\large Lepton + 2 Jets/Leptons }

  \begin{columns}
   \begin{column}{6cm}
    \myCenterBox{Electron channel}
    \includegraphics[width=6cm]{leptonsPlusJets/fig_04a.png}\\
   \end{column}
   \begin{column}{6cm}
    \myCenterBox{Muon channel}
    \includegraphics[width=6cm]{leptonsPlusJets/fig_04b.png}\\
   \end{column}
  \end{columns}
  \begin{columns}
   \begin{column}{6cm}
   \begin{itemize}
  \item At least \textbf{one lepton} with $p_\mathrm{T} >$ 100 GeV.
  \item \textbf{Two additional objects} (leptons or jets) with $p_\mathrm{T} >$ 100 GeV.
  \item Search variable: $\sum p_\mathrm{T}$ of all leptons/jets with $p_\mathrm{T} >$ 60 GeV ($\sum p_\mathrm{T} >$ 2 or 3 TeV).
 \end{itemize}
  \myCenterBox[yellow]{\href{https://cds.cern.ch/record/2139640}{ATLAS-CONF-2016-006}}
   \end{column}
   \begin{column}{6cm}
    \begin{itemize}
     \item The $t\overline{t}$, $W$+jets and $Z$+jets background normalisation are taken from 
	   simultaneous fit in background dedicated CRs.
     \item Validation Region: \\
           1.5 $< \sum p_\mathrm{T} <$ 2.0 TeV.
     \item No significant excess observed.
    \end{itemize}
   
   \end{column}
  \end{columns}

\end{frame}
%*****************************************************************************
%*****************************************************************************
\begin{frame}{\large Lepton + 2 Jets/Leptons}
 
\begin{columns}
 \begin{column}{6cm}
    \begin{itemize}
      \item Limits for rotating black holes with \\2,4 and 6 extra dimensions are set as a function:
      \vspace{0.2cm}
      \begin{itemize}
      \item fundamental Planck scale ($M_D$)
      \vspace{0.2cm}
      \item mass threshold ($M_{th}$)
      \end{itemize}
      \vspace{0.2cm}
      \item  Upper limits on the possible contribution of new physics processes 
      in this class of final states are set at:
      \begin{itemize}
       \item 12.1 fb for $\sum p_\mathrm{T} >$ 2 TeV
       \item 3.4 fb for $\sum p_\mathrm{T} >$ 3 TeV
      \end{itemize}
      \vspace{0.2cm}
      \item \textbf{Limit in $M_{th}$ is improved by almost 3 TeV.}
      
    \end{itemize}
 \end{column}
 \begin{column}{6cm}
  \includegraphics[width=6cm]{leptonsPlusJets/fig_05.png}\\
  \myCenterBox[yellow]{\href{https://cds.cern.ch/record/2139640}{ATLAS-CONF-2016-006}}
 \end{column}
\end{columns}
\end{frame}
%*****************************************************************************
%*****************************************************************************
\begin{frame}{\large Lepton-jet resonances {\small (2 leptons and 2+ jets)}}

  \begin{columns}
   \begin{column}{6cm}
    \myCenterBox{Electron channel}
    \includegraphics[width=5cm]{leptoquarck/eejj_m_LQ_min.pdf}\\
    \begin{itemize}
     \item Inclusive search for a new physics signature of lepton-jet resonances.
     \item {\bf 2 electrons} ({\bf muons}) with $p_\mathrm{T}>30(40)$ GeV and {\bf 2+ jets} with $p_\mathrm{T}>50$ GeV.
     \item 2 lepton-jet pairs: invariant mass difference between them has to be the smallest.
    \end{itemize}
   \myCenterBox[yellow]{\href{http://arxiv.org/abs/1605.06035}{arXiv:1605.06035}}
   \end{column}
   \begin{column}{6cm}
    \myCenterBox{Muon channel}
    \includegraphics[width=5cm]{leptoquarck/mumujj_m_LQ_min.pdf}\\
    \begin{itemize}
%      \item Discriminant variables:
%       \begin{itemize}
%        \item di-lepton invariant mass $m_{\Plepton\Plepton} >$ 130 GeV
%        \item $S_\mathrm{T} = \sum p_\mathrm{T}$ of 2 leptons and 2 leading jets ($>$ 600GeV)
%       \end{itemize}
    \item The $t\overline{t}$ and DY+jets background normalisation are taken from 
	   simultaneous fit in background CRs.
    \vspace{0.2cm}
     \item Search variable:
     \begin{itemize}
      \item $m^{min}_{LQ}$ - minimum invariant mass of 2 lepton-jet pairs\\ {\small ($\sum p_\mathrm{T}$ $>$ 600 GeV; $m_{\Plepton\Plepton}$ $>$ 130 GeV)}.
     \end{itemize}

    \end{itemize}

   \end{column}
  \end{columns}
\end{frame}
%*****************************************************************************
%*****************************************************************************
\begin{frame}{\large Lepton-jet resonances {\small (2 leptons and 2+ jets)}}

  \begin{columns}
   \begin{column}{7.5cm}
    \begin{itemize}
      \item Tested model: pair production of first- and second-generation scalar leptoquarks (mBRW model).
      \vspace{0.15cm}
      \item Excluded LQ’s mass ranges (BR=1):
      \begin{itemize}
       \item \textbf{ $m_{LQ1} < 1100$ GeV; $m_{LQ2} < 1050$ GeV }
      \end{itemize}
      \item Observed limit is \textbf{stronger by 50 GeV} comparing with ATLAS Run 1 result. 
      \vspace{0.15cm}

      \item Limits with different values of BR were set as well.
    \end{itemize}
    \myCenterBox[yellow]{\href{http://arxiv.org/abs/1605.06035}{arXiv:1605.06035}}
    \vspace{0.05cm}
    {\centering
      \includegraphics[width=7.5cm]{leptoquarck/feynman.png}\\
      \includegraphics[width=3.5cm]{leptoquarck/LQ_decay.png}\\
    }

   \end{column}
   \begin{column}{4.5cm}
    \includegraphics[width=4.5cm]{leptoquarck/Limit_eejj.pdf}\\
    \includegraphics[width=4.5cm]{leptoquarck/Limit_mumujj.pdf}\\
   \end{column}
  \end{columns}
\end{frame}
%*****************************************************************************
%*****************************************************************************
\begin{frame}{\large Summary of exclusion limits}

%   \begin{columns}
%   \begin{column}{3cm}
%     \mySmallCenterBox{}
%   \end{column}
%   \begin{column}{3cm}
%     \begin{itemize}
%      \item
%     \end{itemize}
%   \end{column}
%   \begin{column}{3.3cm}
%     \begin{itemize}
%      \item[] 
%     \end{itemize}
%   \end{column}
%   \begin{column}{2.7cm}
%     \begin{itemize}
%      \item[] 
%     \end{itemize}
%   \end{column}
%  \end{columns}

  \begin{columns}
  \begin{column}{3cm}
%     \mySmallCenterBox{}
  \end{column}
  \begin{column}{3cm}
    \begin{itemize}
     \item[] \textbf{Model}
    \end{itemize}
  \end{column}
  \begin{column}{3.3cm}
    \begin{itemize}
     \item[] \textbf{Limit 8 / 13 TeV}
    \end{itemize}
  \end{column}
  \begin{column}{2.7cm}
    \begin{itemize}
     \item[] \textbf{Limit improvement}
    \end{itemize}
  \end{column}
 \end{columns}

%  \vspace{0.1cm}

  \begin{columns}
  \begin{column}{3cm}
    \mySmallCenterBox{Contact Interaction}
  \end{column}
  \begin{column}{3cm}
    \begin{itemize}
     \item[\XSolid] CI $qqqq$
     \item[\XSolid] CI $qq\Plepton\Plepton$
    \end{itemize}
  \end{column}
  \begin{column}{3.3cm}
    \begin{itemize}
     \item[] $\Lambda$: 12.0 / 17.5 TeV
     \item[] $\Lambda$: 21.6 / 23.1 TeV
    \end{itemize}
  \end{column}
  \begin{column}{2.7cm}
    \begin{itemize}
     \item[] 5.5 TeV
     \item[] 1.5 TeV
    \end{itemize}
  \end{column}
 \end{columns}
 
 \vspace{0.2cm}
 
 \begin{columns}
  \begin{column}{3cm}
    \mySmallCenterBox{Extra dimentions}
  \end{column}
  \begin{column}{3cm}
    \begin{itemize}
     \item[\EightStarBold] ADD QBH
     \item[\EightStarBold] ADD BH high $\sum p_\mathrm{T}$
     \item[\EightStarBold] ADD BH multijet
    \end{itemize}
  \end{column}
  \begin{column}{3.3cm}
    \begin{itemize}
     \item[] $M_{th}:$ 5.82 / 8.30 TeV
     \item[] $M_{th}:$ 5.80 / 8.20 TeV
     \item[] $M_{th}:$ 5.80 / 9.55 TeV
    \end{itemize}
  \end{column}
  \begin{column}{2.7cm}
    \begin{itemize}
     \item[] 2.48 TeV
     \item[] 2.40 TeV
     \item[] 3.75 TeV
    \end{itemize}
  \end{column}
 \end{columns}

 \vspace{0.2cm}
 
   \begin{columns}
  \begin{column}{3cm}
    \mySmallCenterBox{Excited fermions}
  \end{column}
  \begin{column}{3cm}
    \begin{itemize}
     \item[\SixteenStarLight] $q^{*} \to qg$
     \item[\SixteenStarLight] $b^{*} \to bg$
    \end{itemize}
  \end{column}
  \begin{column}{3.3cm}
    \begin{itemize}
     \item[] $m_{q^{*}}$: 4.09 / 5.20 TeV
     \item[] $m_{b^{*}}$: -    / 2.10 TeV
    \end{itemize}
  \end{column}
  \begin{column}{2.7cm}
    \begin{itemize}
     \item[] 1.11 TeV
     \item[] - TeV
    \end{itemize}
  \end{column}
 \end{columns}
 
 \vspace{0.2cm}
 
  \begin{columns}
  \begin{column}{3cm}
    \mySmallCenterBox{Gauge bosons}
  \end{column}
  \begin{column}{3cm}
    \begin{itemize}
     \item[\EightAsterisk] SSM $\PZprime \to \Plepton\Plepton$
     \item[\EightAsterisk] SSM $\PZprime \to e^{\pm}\mu^{\mp}$
     \item[\EightAsterisk] SSM $\PWprime \to \Plepton\nu$
     \item[\EightAsterisk] Leptophobic $\PZprime \to bb$
    \end{itemize}
  \end{column}
  \begin{column}{3.3cm}
    \begin{itemize}
     \item[] $M_{\PZprime}:$ 2.90 / 3.40 TeV
     \item[] $M_{\PZprime}:$ 2.50 / 3.01 TeV
     \item[] $M_{\PWprime}:$ 3.24 / 4.07 TeV
     \item[] $M_{\PWprime}:$ -    / 1.5 TeV
    \end{itemize}
  \end{column}
  \begin{column}{2.7cm}
    \begin{itemize}
     \item[] 0.50 TeV
     \vspace{0.09cm}
     \item[] 0.51 TeV
     \vspace{0.09cm}
     \item[] 0.83 TeV
     \vspace{0.09cm}
     \item[] - TeV
    \end{itemize}
  \end{column}
 \end{columns}
 
 \vspace{0.2cm}
 
  \begin{columns}
  \begin{column}{3cm}
    \mySmallCenterBox{Leptoquarks}
  \end{column}
  \begin{column}{3cm}
    \begin{itemize}
     \item[\JackStar] Scalar LQ 1$^{st}$ gen
     \item[\JackStar] Scalar LQ 2$^{nd}$ gen
    \end{itemize}
  \end{column}
  \begin{column}{3.3cm}
    \begin{itemize}
     \item[] $m_{LQ}$: 1.05 / 1.10 TeV
     \item[] $m_{LQ}$: 1.00 / 1.05 TeV
    \end{itemize}
  \end{column}
  \begin{column}{2.7cm}
    \begin{itemize}
     \item[] 0.05 TeV
     \item[] 0.05 TeV
    \end{itemize}
  \end{column}
 \end{columns}
 
 
 \begin{tikzpicture}[overlay]

    %% HELPER draw advanced helping grid with axises:
%     \draw (0,0) to[grid with coordinates] (13,10);

    \draw[red,thick] (6.0,5.0) rectangle (10.75,7.6);
    \draw[yellow,thick] (6.0,1.3) rectangle (10.75,4.7);
    \draw[gray,thick] (6.0,0.05) rectangle (10.75,1.0);
    
  \end{tikzpicture}
 
 
    


\end{frame}
%*****************************************************************************
%*****************************************************************************
\begin{frame}{\large Conclusions}
 \begin{itemize}
  \item[\Checkmark] Current results on searches for new physics in high-mass fermionic final states and 
	jets have been presented.
  \vspace{0.3cm}
  \item[\Checkmark] The sensitivity of all searches already exceeds Run 1 results and 
	new mass regions have been explored. 
  \vspace{0.3cm}
  \item[\Checkmark] No significant excesses are observed but stronger exclusion limits are set for various models
beyond the Standard Model.
  \vspace{0.3cm}
  \item The expected ten times higher luminosity in 2016 will allow a 
	deeper exploration of the 13 TeV regime.
  \vspace{0.3cm}
 \end{itemize}

\end{frame}
%*****************************************************************************
\backupbegin
%*****************************************************************************
\begin{frame}{\large BACKUP}
 
\end{frame}
%*****************************************************************************
%*****************************************************************************
\begin{frame}{\large Multijet}
 \begin{itemize}
  \item At least three jets with $p_\mathrm{T} >$ 50 GeV and $|\eta| <$ 2.8
  \item Search variable: Scalar sum of jet transverse momenta ($H_\mathrm{T}$) $>$ 1 TeV
  \item SM background predicted by data-driven fit-based technique based on 3 regions:
  \begin{itemize}
   \item Control region (CR) - fit background model to data (10 functions tested)
   \item Validation region (VR) - cross-check that the extrapolation procedure is working properly
   \item Signal region (SR) - use fit to estimate amount of the SM background
  \end{itemize}
  \item In order to be sure that CR and VR are not contaminated by possible signal - 
	use bootstrap approach:
  \begin{itemize}
   \item examine data sets whose size increases by approximately a factor of ten at each step, 
   starting with a sample whose sensitivity is slightly beyond the Run 1 limit.
   \item if a search in one step sees no new physics, the possible contributions of signal to 
   the control and validation regions of the next step are small.
  \end{itemize}
 \end{itemize}
 
 \begin{columns}
  \begin{column}{3cm}
   \myCenterBox{Step 1: 6.5 pb$^{-1}$}
   \includegraphics[width=3cm]{multijet/fig_05.png}\\
  \end{column}
  \begin{column}{3cm}
  \myCenterBox{Step 2: 74 pb$^{-1}$}
   \includegraphics[width=3cm]{multijet/fig_06.png}\\
  \end{column}
  \begin{column}{3cm}
  \myCenterBox{Step 3: 0.44 fb$^{-1}$}
   \includegraphics[width=3cm]{multijet/fig_07.png}\\
  \end{column}
  \begin{column}{3cm}
  \myCenterBox{Step 4: 3.0 fb$^{-1}$}
   \includegraphics[width=3cm]{multijet/fig_08.png}\\
  \end{column}
 \end{columns}

\end{frame}
%*****************************************************************************
% %*****************************************************************************
% \begin{frame}{\large Multijet: String Ball Model}
%   \begin{columns}
%   \begin{column}{6cm}
%       \begin{itemize}
%       \item Limits for String balls with 6 extra dimensions are set as a function:
%       \begin{itemize}
%       \item string scale ($M_S$)
%       \item mass threshold ($M_{th}$)
%       \item string coupling ($g_S$)
%       \end{itemize}
%       \item A string ball can exit in 3 different phases depending on $M_S$, $g_S$, and its mass $M_{th}$.  
% %       You are suppose to see 3 different curves on this plot and 2 different transitions points.  
%       \begin{itemize}
%        \item The high $M_{th}$ is when the string ball is a black hole, 
%        \item the middle regions is the stringy state, 
%        \item and the low $M_{th}$ region is the unitary state.
%       \end{itemize}
%    \end{itemize}
%   \end{column}
%   \begin{column}{6cm}
%    \myCenterBox{String ball model}
%    \includegraphics[width=6cm]{multijet/fig_11b.png}\\
%    \end{column}
%  \end{columns}
% 
%  \vspace{0.5cm}
%  \myCenterBox[yellow]{\href{http://arxiv.org/abs/1512.02586}{arXiv:1512.02586}}
%  
% \end{frame}
% %*****************************************************************************
%*****************************************************************************
\begin{frame}{\large Lepton-jet resonances}

  \begin{columns}
   \begin{column}{7.5cm}
    \begin{itemize}
      \item Tested model: pair production of first- and second-generation scalar leptoquarks (mBRW model).
      \vspace{0.15cm}
      \item Excluded LQ’s mass ranges (BR=1):
      \begin{itemize}
       \item $m_{LQ1} < 1100$ GeV
       \item $m_{LQ2} < 1050$ GeV
      \end{itemize}
      \vspace{0.15cm}
      \item Observed limit is \textbf{stronger by 50 GeV} comparing with ATLAS Run 1 result. 
    \end{itemize}
    \myCenterBox[yellow]{\href{http://arxiv.org/abs/1605.06035}{arXiv:1605.06035}}
    \vspace{0.05cm}
    {\centering
      \includegraphics[width=7.5cm]{leptoquarck/feynman.png}\\
      \includegraphics[width=3.5cm]{leptoquarck/LQ_decay.png}\\
    }

   \end{column}
   \begin{column}{4.5cm}
    \myVerySmallCenterBox{Limits for different assumptions on leptoquark BR}
    \includegraphics[width=4.15cm]{leptoquarck/Limit_eejj_beta.pdf}\\
    \includegraphics[width=4.15cm]{leptoquarck/Limit_mumujj_beta.pdf}\\
   \end{column}
  \end{columns}
\end{frame}
%*****************************************************************************
%*****************************************************************************
\begin{frame}{\large Lepton + 2 Jets/Leptons}

\myCenterBox{Motivation of analysis selection wrt QBH models}

\begin{itemize}
 \item Gravitational interaction couples to the energy-momentum tensor rather than gauge
quantum numbers, final states are expected to be populated "democratically".
 \item It is expected that a significant fraction of final states will contain leptons.
 \item Lepton + 2 Jets/Leptons selection enhance the signal strength compared
to the dominant background at the LHC which arises from quark and gluon scattering processes forming
hadronic final states.
\end{itemize}

\end{frame}
%*****************************************************************************
%*****************************************************************************
\begin{frame}{\large Lepton + 2 Jets/Leptons}
 \myCenterBox{Normalization Control Regions}
\begin{columns}
 \begin{column}{4cm}
  \includegraphics[width=4cm]{leptonsPlusJets/fig_01a.png}\\
 \end{column}
 \begin{column}{4cm}
  \includegraphics[width=4cm]{leptonsPlusJets/fig_02a.png}\\
 \end{column}
 \begin{column}{4cm}
  \includegraphics[width=4cm]{leptonsPlusJets/fig_03a.png}\\
 \end{column}
\end{columns}
\begin{itemize}
 \item The backgrounds predictions were adjusted by the likelihood fit.
 \item Scale factors:
 \begin{itemize}
  \item $W+$jets: 0.81 $\pm$ 0.07 
  \item $Z+$jets: 1.01 $\pm$ 0.08
  \item $t\overline{t}$: 0.95 $\pm$ 0.08
 \end{itemize}

\end{itemize}

\end{frame}
%*****************************************************************************
%*****************************************************************************
% \bgroup
% \setbeamercolor{background canvas}{bg=white}
\begin{frame}{}

    \begin{tikzpicture}[overlay]

    %% HELPER draw advanced helping grid with axises:
%     \draw (0,-5) to[grid with coordinates] (11,3);

    \node[right] (textNode) at (3,0) {
      { \large \bf Theoretical BSM models}
    };
    
    \node[right] (n11) at (5,-0.7) {
        \EightStarTaper $\PWprime$ and $\PZprime$ (SSM, E6 GUT)
    };
    
    \node[right] (n21) at (5,-1.2) {
        \EightStarTaper Higher-dimensional QBH
    };
    
    \node[right] (n31) at (5,-1.7) {
        \EightStarTaper String balls
    };
    
    \node[right] (n41) at (5,-2.2) {
        \EightStarTaper Contact Interaction
    };
    
    \node[right] (n51) at (5,-2.7) {
        \EightStarTaper Excited quarks
    };
    
    \node[right] (n61) at (5,-3.2) {
        \EightStarTaper Leptoquarks
    };
    
    \tikz[overlay]\draw[thick,black,->] ([xshift=-0.5cm]textNode.south) to [out=270, in=180] ([xshift=-0.1pt]n11.west);
    \tikz[overlay]\draw[thick,black,->] ([xshift=-0.75cm]textNode.south) to [out=270, in=180] ([xshift=-0.1pt]n21.west);
    \tikz[overlay]\draw[thick,black,->] ([xshift=-1.0cm]textNode.south) to [out=270, in=180] ([xshift=-0.1pt]n31.west);
    \tikz[overlay]\draw[thick,black,->] ([xshift=-1.25cm]textNode.south) to [out=270, in=180] ([xshift=-0.1pt]n41.west);
    \tikz[overlay]\draw[thick,black,->] ([xshift=-1.5cm]textNode.south) to [out=270, in=180] ([xshift=-0.1pt]n51.west);
    \tikz[overlay]\draw[thick,black,->] ([xshift=-1.75cm]textNode.south) to [out=270, in=180] ([xshift=-0.1pt]n61.west);

    \end{tikzpicture}

\end{frame}
% \egroup
%*****************************************************************************
%*****************************************************************************
\begin{frame}{\large $\PWprime$ and $\PZprime$ models}
\begin{itemize}
 \item \textbf{Sequential Standard Model (SSM)}
  \begin{itemize}
    \item $\PWprime$ and $\PZprime$ have the same couplings to fermions as the Standard Model (SM) W and Z bosons.
  \end{itemize}
 \item \textbf{$E_6$ Grand Unified Theory}
  \begin{itemize}
    \item $E_6 \rightarrow SU(5) \times U(1)_{\chi} \times U(1)_{\psi}$
    \item $\PZprime(\theta_{E_6}) = Z'_{\psi}\cos\theta_{E_6} + Z'_{\chi}\sin\theta_{E_6}$
    \item 6 commonly motivated states of $\PZprime$ namely:
	  $Z'_{\psi}, Z'_{N}, Z'_{\eta}, Z'_{I}, Z'_{S}, Z'_{\chi}$
  \end{itemize}
 \item \textbf{Leptophobic $\PZprime$}
 \begin{itemize}
  \item $\PZprime$ boson does not couple to the leptons.
  \item Derived from different scenarios of the E6 GUT.
 \end{itemize}
 
 
\end{itemize}
  
\begin{columns}
 \begin{column}{6cm}
  \includegraphics[width=6cm]{bTaggedDijet/Zprime.png}\\
 \end{column}
 \begin{column}{6cm}
  \includegraphics[width=5cm]{bTaggedDijet/Zprime_lep.png}\\
 \end{column}
\end{columns}

\end{frame}
%*****************************************************************************
%*****************************************************************************
\begin{frame}{\large Higher-dimensional QBH}
\begin{columns}
 \begin{column}{6cm}
  \begin{itemize}
   \item ADD model - proposed by Arkani-Hamed, Dimopoulos and Dvali.
   \item RS model - proposed by Randall and Sundrum.
  \end{itemize}
  \begin{itemize}
   \item fundamental Planck scale ($M_D$) - energy scale at which quantum effects of gravity become strong.
   \item mass threshold ($M_{th}$) - threshold where black holes start to form.
  \end{itemize}
  \begin{itemize}
   \item Ideally, black holes would decay isotropically to many energetic particles, in keeping with the prediction of thermal Hawking radiation.
   \item Expected black hole signature is low multiplicity final states {\small (ADD n = 6)}:
   \begin{equation}
    \langle N \rangle \sim \left( \dfrac{M_{BH}}{M_{D}} \right)^{8/7}
   \end{equation}

  \end{itemize}


 \end{column}
 \begin{column}{6cm}
 
 {\centering
  \textbf{QBH signal on top of SM QCD backgound}
 }
 \vspace{0.4cm}\\
 
  \includegraphics[width=6cm]{schematicBlackHoleDistribution.png}\\
  
  \begin{tikzpicture}[overlay]

    %% HELPER draw advanced helping grid with axises:
%     \draw (0,0) to[grid with coordinates] (7,4);
      
    \node[right] (textNode) at (2.2,3.5) {
      \myVerySmallCenterBox{ $d\sigma/dM_{jj} (pb/GeV)$ vs. $M_{jj} (TeV)$ }
    };
    \node[right] (textNode) at (1,1) {
      {\tiny \href{http://arxiv.org/abs/0708.3017}{arXiv:0708.3017} }
    };
        
  
  
    \end{tikzpicture}
  
 \end{column}
\end{columns}

\end{frame}
%*****************************************************************************
%*****************************************************************************
\begin{frame}{\large String balls}

In string theory, black holes have a minimum mass below which they
transition into highly excited long and jagged strings  --- ``string
balls''.
\vspace{0.2cm}

In summary, the elementary (parton) cross section for string ball/BH
production is
\begin{eqnarray}
\sigma\sim\left\{
\begin{array}{ll}
\displaystyle 
{g_s^2 M_{SB}^2\over M_s^4} &\qquad M_s\ll M_{SB}\leq M_s/ g_s\,,\\
\displaystyle 
{1\over M_s^{2}}&\qquad M_s/ g_s< M_{SB}\leq M_s/ g_s^2\,,\\
\displaystyle
{1\over M_P^{2}}\left({M_{BH}\over M_P}\right)^{2\over n+1} 
&\qquad M_s/ g_s^2<M_{BH}\,.
\end{array}
\right.
\end{eqnarray}
$M_{SB}$ ($M_{BH}$) is the mass of the string ball (black hole), and we
have used $\alpha'=M_s^{-2}$.

\textbf{
The first two mass ranges lead to string balls, the third to black
holes. }

\vspace{0.1cm}
{\centering
 \includegraphics[width=5.3cm]{multijet/fig_11b.png}\\
}

\begin{tikzpicture}[overlay]

    %% HELPER draw advanced helping grid with axises:
%     \draw (0,0) to[grid with coordinates] (12,9);
      
    \node[right] (textNode) at (8.5,7.5) {
      \myVerySmallCenterBox{\href{http://arxiv.org/abs/hep-ph/0108060}{arXiv:hep-ph/0108060}}
    };

    \end{tikzpicture}

\end{frame}
%*****************************************************************************
%*****************************************************************************
\begin{frame}{\large Contact Interaction}
  
  \myCenterBox{\textbf{Non-Resonant} Contact Interaction (CI)}
  \begin{itemize}
    \item Quark and Lepton Compositeness:
    \begin{itemize}
      \item Characteristic energy scale $\varLambda$ corresponds to binding energy between constituents
      \item $\eta_{ij}$ gives chiral structure of the interaction
    \end{itemize}
    \item Excpect non-resonant deviations in the di-lepton mass spectrum.
  \end{itemize}

{\centering
  \includegraphics[width=7cm]{dilepton/CI_feynman.png}\\
}

CI Lagrangian is used to describe a new interaction or compositeness in the process $q\overline{q} \to \ell^+\ell^-$:

\begin{eqnarray}\label{lagrangian}
\mathcal L & = & \frac{g^2}{\Lambda^2}\;[ \eta_{\rm LL} \, (\overline q_{\rm L}\gamma_{\mu} q_{\rm L})\,(\overline\ell_{\rm L}\gamma^{\mu}\ell_{\rm L})
+\eta_{\rm RR} (\overline q_{\rm R}\gamma_{\mu} q_{\rm R}) \,(\overline\ell_{\rm R}\gamma^{\mu}\ell_{\rm R}) \\ \nonumber
&  & ~~~~~~~~ +\eta_{\rm LR} (\overline q_{\rm L}\gamma_{\mu} q_{\rm L}) \,(\overline\ell_{\rm R}\gamma^{\mu}\ell_{\rm R})
+\eta_{\rm RL} (\overline q_{\rm R}\gamma_{\mu} q_{\rm R}) \,(\overline\ell_{\rm L}\gamma^{\mu}\ell_{\rm L}) ] \:
\end{eqnarray}


\end{frame}
%*****************************************************************************
%*****************************************************************************
\begin{frame}{\large Excited quarks}
  
  \begin{itemize}
   \item Excited quarks are a consequence of quark compositeness models that were proposed to explain the generational structure and mass hierarchy of quarks.

  \end{itemize}

  {\centering
  \includegraphics[width=6cm]{bTaggedDijet/excitedQuark.png}\\
  }

\end{frame}
%*****************************************************************************
%*****************************************************************************
\begin{frame}{\large Leptoquarks}
  
  \begin{itemize}
   \item LQs possess non-zero baryon and lepton numbers; \\their existence would provide a connection between quarks and leptons.
   \item LQs carry a colour-triplet charge and a fractional electric charge.
   \item The signal benchmark model in the analysis is the minimal Buchm\"uller--R\"uckl--Wyler model (mBRW).
   \begin{itemize}
    \item Lepton number and baryon number are separately conserved to prevent fast proton decay.
    \item The LQ couplings are also considered to be purely chiral.
    \item LQs belong to three generations (first, second and third) which interact only with lepton--quark pairs within the same generation.
    \item Lepton-flavour violation is suppressed.
   \end{itemize}
  \end{itemize}

  \begin{columns}
   \begin{column}{4cm}
    \includegraphics[width=4cm]{leptoquarck/fig_02a.png}\\
    \includegraphics[width=4cm]{leptoquarck/fig_02b.png}\\
   \end{column}
   \begin{column}{4cm}
    \includegraphics[width=4cm]{leptoquarck/fig_02c.png}\\
    \includegraphics[width=4cm]{leptoquarck/fig_02d.png}\\
   \end{column}
   \begin{column}{4cm}
    \includegraphics[width=4cm]{leptoquarck/LQ_decay.png}\\
   \end{column}
  \end{columns}

\end{frame}
%*****************************************************************************
%*****************************************************************************
% \bgroup
% \setbeamercolor{background canvas}{bg=white}
\begin{frame}{}

    \begin{tikzpicture}[overlay]

    %% HELPER draw advanced helping grid with axises:
%     \draw (0,-5) to[grid with coordinates] (11,3);

    \node[right] (textNode) at (3,0) {
      { \large \bf Other materials}
    };
    
    \node[right] (n11) at (5,-0.7) {
        \EightStarTaper Summary plot of Exotics searches.
    };
    
    \node[right] (n21) at (5,-1.2) {
        \EightStarTaper QCD background estimation
    };
%     
%     \node[right] (n31) at (5,-1.7) {
%         \EightStarTaper Contact Interaction
%     };
%     
%     \node[right] (n41) at (5,-2.2) {
%         \EightStarTaper Excited quarks
%     };
%     
%     \node[right] (n51) at (5,-2.7) {
%         \EightStarTaper Leptoquarks
%     };
    
    \tikz[overlay]\draw[thick,black,->] ([xshift=-0.5cm]textNode.south) to [out=270, in=180] ([xshift=-0.1pt]n11.west);
    \tikz[overlay]\draw[thick,black,->] ([xshift=-0.75cm]textNode.south) to [out=270, in=180] ([xshift=-0.1pt]n21.west);
%     \tikz[overlay]\draw[thick,black,->] ([xshift=-1.0cm]textNode.south) to [out=270, in=180] ([xshift=-0.1pt]n31.west);
%     \tikz[overlay]\draw[thick,black,->] ([xshift=-1.25cm]textNode.south) to [out=270, in=180] ([xshift=-0.1pt]n41.west);
%     \tikz[overlay]\draw[thick,black,->] ([xshift=-1.5cm]textNode.south) to [out=270, in=180] ([xshift=-0.1pt]n51.west);

    \end{tikzpicture}

\end{frame}
% \egroup
%*****************************************************************************
%*****************************************************************************
\bgroup
\setbeamercolor{background canvas}{bg=white}
\begin{frame}{}

\begin{tikzpicture}[overlay]

    %% HELPER draw advanced helping grid with axises:
    \draw (0,-5) to[grid with coordinates] (11,3);

    \node[right] (textNode) at (-2.5,-1) {
      \includegraphics[width=14cm]{ATLAS_Exotics_Summary.pdf}
    };
  
    \end{tikzpicture}

   

\end{frame}
\egroup
%*****************************************************************************
%*****************************************************************************
 { % all template changes are local to this group.
    \setbeamertemplate{navigation symbols}{}
    \begin{frame}{QCD background estimation}
        \begin{tikzpicture}[remember picture,overlay]
            \node[at=(current page.center)] {
                \includegraphics[width=0.97\paperwidth]{/home/oviazlo/PhD_study/myReports/pictures/Wprime/Oct14_exotics/Yosuke_MatrixMethod.pdf}
            };
%             \node[at=(2.5,3.5)] {
% 	      TEST
%             };
%             \node[anchor=south] at (current page.south) {\vspace{-4cm} I am south-bound...};
%             \draw[thick] (current page.south west) rectangle (current page.north east);
        \end{tikzpicture}
     \end{frame}
}
%*****************************************************************************
%***************************************************************************** 
 \begin{frame}{QCD background estimation}
 
      
  Number of fake leptons coming from QCD background can be expressed as: \\
  \vspace{0.2cm}
  \hspace{0.6cm}
  $\epsilon_F N_F=\frac{\epsilon_{F}}{\epsilon_{R}-\epsilon_{F}}\left(\epsilon_{R}(N_{L}+N_{T})-N_{T}\right)$

  
  \vspace{0.5cm}
  Tight selection = Signal selection \\
  ``Loose'' selection:
  \begin{itemize}
   \item Electron: Signal selection wo isolation cut and looser electron ID
   \item Muon: Signal selection wo isolation cut
  \end{itemize}
  
  \vspace{0.5cm}
  Fake Efficiencies are calculated by using jet enriched control region:
  \begin{itemize}
   \item No Z Candidate ($|m_{ll}-m_{Z}| > 20$ GeV)
   \item Inverted $E_{T}^{miss}$ cut
   \item At least one jet with $p_{T} > 40$ GeV
  \end{itemize}


 \end{frame}
%*****************************************************************************
\backupend
\end{document}

